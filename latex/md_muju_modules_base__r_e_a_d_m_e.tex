Un proyecto que utilice esta plantilla debe contener un archivo {\ttfamily makefile} en la carpeta principal del proyecto, en el cual se deben realizar, como mínimo, las definicion de las siguientes variables

{\ttfamily M\+O\+D\+U\+L\+ES} = define una lista de directorios con el codigo fuente de modulos que seran compilados como librerias estáticas (archivos {\ttfamily .a}) antes de compilar el proyecto principal

{\ttfamily B\+O\+A\+RD} = define el nombre de la placa para la cual se debe compilar el proyecto.

{\ttfamily P\+R\+O\+J\+E\+CT} = define el directorio principal del proyecto que se desea compilar.

\section*{Compilación de los modulos como librerias estáticas}

Para cada directorio definido en la variable {\ttfamily M\+O\+D\+U\+L\+ES} se definen las siguientes variables

{\ttfamily C\+A\+R\+P\+E\+T\+A\+\_\+\+M\+O\+D\+U\+L\+O\+\_\+\+I\+NC} = define la lista de directorios donde deben buscarse los archivos de cabecera {\ttfamily .h}

{\ttfamily C\+A\+R\+P\+E\+T\+A\+\_\+\+M\+O\+D\+U\+L\+O\+\_\+\+S\+RC} = define la lista de directorios donde deben buscarse los archivos fuentes {\ttfamily .c} y {\ttfamily .s} del modulo.

{\ttfamily C\+A\+R\+P\+E\+T\+A\+\_\+\+M\+O\+D\+U\+L\+O\+\_\+\+O\+BJ} = define la lista de archivos objeto {\ttfamily .o} que deben enlazarse para completar el modulo. 