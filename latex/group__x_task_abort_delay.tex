\hypertarget{group__x_task_abort_delay}{}\section{x\+Task\+Abort\+Delay}
\label{group__x_task_abort_delay}\index{x\+Task\+Abort\+Delay@{x\+Task\+Abort\+Delay}}
task. h 
\begin{DoxyPre}
BaseType\_t \hyperlink{externals_2freertos_2include_2task_8h_afefe333df0492c8411c0094badd25185}{xTaskAbortDelay( TaskHandle\_t xTask )};
\end{DoxyPre}


I\+N\+C\+L\+U\+D\+E\+\_\+x\+Task\+Abort\+Delay must be defined as 1 in Free\+R\+T\+O\+S\+Config.\+h for this function to be available.

A task will enter the Blocked state when it is waiting for an event. The event it is waiting for can be a temporal event (waiting for a time), such as when \hyperlink{externals_2freertos_2include_2task_8h_aa154068cecd7f31446a7a84af44ab1a3}{v\+Task\+Delay()} is called, or an event on an object, such as when \hyperlink{vendor_2ceedling_2plugins_2freertos_2src_2freertos_2include_2queue_8h_af1549eac0e7f05694a59a0b967c80be3}{x\+Queue\+Receive()} or \hyperlink{externals_2freertos_2include_2task_8h_a725a2da114ef870747edd7fd19d77bab}{ul\+Task\+Notify\+Take()} is called. If the handle of a task that is in the Blocked state is used in a call to \hyperlink{externals_2freertos_2include_2task_8h_afefe333df0492c8411c0094badd25185}{x\+Task\+Abort\+Delay()} then the task will leave the Blocked state, and return from whichever function call placed the task into the Blocked state.

There is no \textquotesingle{}From\+I\+SR\textquotesingle{} version of this function as an interrupt would need to know which object a task was blocked on in order to know which actions to take. For example, if the task was blocked on a queue the interrupt handler would then need to know if the queue was locked.


\begin{DoxyParams}{Parameters}
{\em x\+Task} & The handle of the task to remove from the Blocked state.\\
\hline
\end{DoxyParams}
\begin{DoxyReturn}{Returns}
If the task referenced by x\+Task was not in the Blocked state then pd\+F\+A\+IL is returned. Otherwise pd\+P\+A\+SS is returned. 
\end{DoxyReturn}
