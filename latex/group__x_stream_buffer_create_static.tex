\hypertarget{group__x_stream_buffer_create_static}{}\section{x\+Stream\+Buffer\+Create\+Static}
\label{group__x_stream_buffer_create_static}\index{x\+Stream\+Buffer\+Create\+Static@{x\+Stream\+Buffer\+Create\+Static}}
\hyperlink{stream__buffer_8h}{stream\+\_\+buffer.\+h}


\begin{DoxyPre}
  StreamBufferHandle\_t xStreamBufferCreateStatic( size\_t xBufferSizeBytes,
                                               size\_t xTriggerLevelBytes,
                                               uint8\_t *pucStreamBufferStorageArea,
                                               StaticStreamBuffer\_t *pxStaticStreamBuffer );
  \end{DoxyPre}
 Creates a new stream buffer using statically allocated memory. See \hyperlink{stream__buffer_8h_a39aa4dd8b83e2df7ded291f863fb5fed}{x\+Stream\+Buffer\+Create()} for a version that uses dynamically allocated memory.

config\+S\+U\+P\+P\+O\+R\+T\+\_\+\+S\+T\+A\+T\+I\+C\+\_\+\+A\+L\+L\+O\+C\+A\+T\+I\+ON must be set to 1 in Free\+R\+T\+O\+S\+Config.\+h for \hyperlink{stream__buffer_8h_a3c248575ac1b83801db605b32a118f77}{x\+Stream\+Buffer\+Create\+Static()} to be available.


\begin{DoxyParams}{Parameters}
{\em x\+Buffer\+Size\+Bytes} & The size, in bytes, of the buffer pointed to by the puc\+Stream\+Buffer\+Storage\+Area parameter.\\
\hline
{\em x\+Trigger\+Level\+Bytes} & The number of bytes that must be in the stream buffer before a task that is blocked on the stream buffer to wait for data is moved out of the blocked state. For example, if a task is blocked on a read of an empty stream buffer that has a trigger level of 1 then the task will be unblocked when a single byte is written to the buffer or the task\textquotesingle{}s block time expires. As another example, if a task is blocked on a read of an empty stream buffer that has a trigger level of 10 then the task will not be unblocked until the stream buffer contains at least 10 bytes or the task\textquotesingle{}s block time expires. If a reading task\textquotesingle{}s block time expires before the trigger level is reached then the task will still receive however many bytes are actually available. Setting a trigger level of 0 will result in a trigger level of 1 being used. It is not valid to specify a trigger level that is greater than the buffer size.\\
\hline
{\em puc\+Stream\+Buffer\+Storage\+Area} & Must point to a uint8\+\_\+t array that is at least x\+Buffer\+Size\+Bytes + 1 big. This is the array to which streams are copied when they are written to the stream buffer.\\
\hline
{\em px\+Static\+Stream\+Buffer} & Must point to a variable of type Static\+Stream\+Buffer\+\_\+t, which will be used to hold the stream buffer\textquotesingle{}s data structure.\\
\hline
\end{DoxyParams}
\begin{DoxyReturn}{Returns}
If the stream buffer is created successfully then a handle to the created stream buffer is returned. If either puc\+Stream\+Buffer\+Storage\+Area or px\+Staticstream\+Buffer are N\+U\+LL then N\+U\+LL is returned.
\end{DoxyReturn}
Example use\+: 
\begin{DoxyPre}\end{DoxyPre}



\begin{DoxyPre}  // Used to dimension the array used to hold the streams.  The available space
  // will actually be one less than this, so 999.
#define STORAGE\_SIZE\_BYTES 1000\end{DoxyPre}



\begin{DoxyPre}  // Defines the memory that will actually hold the streams within the stream
  // buffer.
  static uint8\_t ucStorageBuffer[ STORAGE\_SIZE\_BYTES ];\end{DoxyPre}



\begin{DoxyPre}  // The variable used to hold the stream buffer structure.
  StaticStreamBuffer\_t xStreamBufferStruct;\end{DoxyPre}



\begin{DoxyPre}  void MyFunction( void )
  \{
  StreamBufferHandle\_t xStreamBuffer;
  const size\_t xTriggerLevel = 1;\end{DoxyPre}



\begin{DoxyPre}   xStreamBuffer = xStreamBufferCreateStatic( sizeof( ucBufferStorage ),
                                              xTriggerLevel,
                                              ucBufferStorage,
                                              \&xStreamBufferStruct );\end{DoxyPre}



\begin{DoxyPre}   // As neither the pucStreamBufferStorageArea or pxStaticStreamBuffer
   // parameters were NULL, xStreamBuffer will not be NULL, and can be used to
   // reference the created stream buffer in other stream buffer API calls.\end{DoxyPre}



\begin{DoxyPre}   // Other code that uses the stream buffer can go here.
  \}\end{DoxyPre}



\begin{DoxyPre}  \end{DoxyPre}
 