\subsection*{Overview}

The module\+\_\+generator plugin adds a pair of new commands to Ceedling, allowing you to make or remove modules according to predefined templates. W\+Ith a single call, Ceedling can generate a source, header, and test file for a new module. If given a pattern, it can even create a series of submodules to support specific design patterns. Finally, it can just as easily remove related modules, avoiding the need to delete each individually.

Let\textquotesingle{}s say, for example, that you want to create a single module named {\ttfamily Mad\+Science}.


\begin{DoxyCode}
ceedling module:create[MadScience]
\end{DoxyCode}


It says we\textquotesingle{}re speaking to the module plugin, and we want to create a new module. The name of that module is between the brackets. It will keep this case, unless you have specified a different default (see configuration). It will create three files\+: {\ttfamily Mad\+Science.\+c}, {\ttfamily Mad\+Science.\+h}, and {\ttfamily Test\+Mad\+Science.\+c}. {\itshape N\+O\+TE} that it is important that there are no spaces between the brackets. We know, it\textquotesingle{}s annoying... but it\textquotesingle{}s the rules.

You can also create an entire pattern of files. To do that, just add a second argument to the pattern ID. Something like this\+:


\begin{DoxyCode}
ceedling module:create[SecretLair,mch]
\end{DoxyCode}


In this example, we\textquotesingle{}d create 9 files total\+: 3 headers, 3 source files, and 3 test files. These files would be named {\ttfamily Secret\+Lair\+Model}, {\ttfamily Secret\+Lair\+Conductor}, and {\ttfamily Secret\+Lair\+Hardware}. Isn\textquotesingle{}t that nice?

Similarly, you can create stubs for all functions in a header file just by making a single call to your handy {\ttfamily stub} feature, like this\+:


\begin{DoxyCode}
ceedling module:stub[SecretLair]
\end{DoxyCode}


This call will look in Secret\+Lair.\+h and will generate a file Secret\+Lair.\+c that contains a stub for each function declared in the header! Even better, if Secret\+Lair.\+c already exists, it will add only new functions, leaving your existing calls alone so that it doesn\textquotesingle{}t cause any problems.

\subsection*{Configuration}

Enable the plugin in your project.\+yml by adding {\ttfamily module\+\_\+generator} to the list of enabled plugins.

Then, like much of Ceedling, you can just run as-\/is with the defaults, or you can override those defaults for your own needs. For example, new source and header files will be automatically placed in the {\ttfamily src/} folder while tests will go in the {\ttfamily test/} folder. That\textquotesingle{}s great if your project follows the default ceedling structure... but what if you have a different structure?


\begin{DoxyCode}
:module\_generator:
  :project\_root: ./
  :source\_root: source/
  :inc\_root: includes/
  :test\_root: tests/
\end{DoxyCode}


Now I\textquotesingle{}ve redirected the location where modules are going to be generated.

\subsubsection*{Includes}

You can make it so that all of your files are generated with a standard include list. This is done by adding to the {\ttfamily \+:includes} array. For example\+:


\begin{DoxyCode}
:module\_generator:
  :includes:
    :tst:
      - defs.h
      - board.h
    :src:
      - board.h
\end{DoxyCode}


\subsubsection*{Boilerplates}

You can specify the actual boilerplate used for each of your files. This is the handy place to put that corporate copyright notice (or maybe a copyleft notice, if that\textquotesingle{}s your perference?)


\begin{DoxyCode}
:module\_generator:
  :boilerplates: |
    /***************************
    * This file is Awesome.    *
    * That is All.             *
    ***************************/
\end{DoxyCode}


\subsubsection*{Test Defines}

You can specify the \char`\"{}\#ifdef T\+E\+S\+T\char`\"{} at the top of the test files with a custom define. This example will put a \char`\"{}\#ifdef C\+E\+E\+D\+L\+I\+N\+G\+\_\+\+T\+E\+S\+T\char`\"{} at the top of the test files.


\begin{DoxyCode}
:module\_generator:
  :test\_define: CEEDLING\_TEST
\end{DoxyCode}


\subsubsection*{Naming Convention}

Finally, you can force a particular naming convention. Even if someone calls the generator with something like {\ttfamily My\+New\+Module}, if they have the naming convention set to {\ttfamily \+:caps}, it will generate files like {\ttfamily M\+Y\+\_\+\+N\+E\+W\+\_\+\+M\+O\+D\+U\+L\+E.\+c}. This keeps everyone on your team behaving the same way.

Your options are as follows\+:


\begin{DoxyItemize}
\item {\ttfamily \+:bumpy} -\/ Bumpy\+Files\+Looks\+Like\+So
\item {\ttfamily \+:camel} -\/ camel\+Files\+Are\+Similar\+But\+Start\+Low
\item {\ttfamily \+:snake} -\/ snake\+\_\+case\+\_\+is\+\_\+all\+\_\+lower\+\_\+and\+\_\+uses\+\_\+underscores
\item {\ttfamily \+:caps} -\/ C\+A\+P\+S\+\_\+\+F\+E\+E\+L\+S\+\_\+\+L\+I\+K\+E\+\_\+\+Y\+O\+U\+\_\+\+A\+R\+E\+\_\+\+S\+C\+R\+E\+A\+M\+I\+NG 
\end{DoxyItemize}