\hypertarget{group__x_message_buffer_create_static}{}\section{x\+Message\+Buffer\+Create\+Static}
\label{group__x_message_buffer_create_static}\index{x\+Message\+Buffer\+Create\+Static@{x\+Message\+Buffer\+Create\+Static}}
\hyperlink{message__buffer_8h}{message\+\_\+buffer.\+h}


\begin{DoxyPre}
  MessageBufferHandle\_t xMessageBufferCreateStatic( size\_t xBufferSizeBytes,
                                                 uint8\_t *pucMessageBufferStorageArea,
                                                 StaticMessageBuffer\_t *pxStaticMessageBuffer );
  \end{DoxyPre}
 Creates a new message buffer using statically allocated memory. See \hyperlink{message__buffer_8h_a2959cd0e3d2bd20d46908e5c9872be36}{x\+Message\+Buffer\+Create()} for a version that uses dynamically allocated memory.


\begin{DoxyParams}{Parameters}
{\em x\+Buffer\+Size\+Bytes} & The size, in bytes, of the buffer pointed to by the puc\+Message\+Buffer\+Storage\+Area parameter. When a message is written to the message buffer an additional sizeof( size\+\_\+t ) bytes are also written to store the message\textquotesingle{}s length. sizeof( size\+\_\+t ) is typically 4 bytes on a 32-\/bit architecture, so on most 32-\/bit architecture a 10 byte message will take up 14 bytes of message buffer space. The maximum number of bytes that can be stored in the message buffer is actually (x\+Buffer\+Size\+Bytes -\/ 1).\\
\hline
{\em puc\+Message\+Buffer\+Storage\+Area} & Must point to a uint8\+\_\+t array that is at least x\+Buffer\+Size\+Bytes + 1 big. This is the array to which messages are copied when they are written to the message buffer.\\
\hline
{\em px\+Static\+Message\+Buffer} & Must point to a variable of type Static\+Message\+Buffer\+\_\+t, which will be used to hold the message buffer\textquotesingle{}s data structure.\\
\hline
\end{DoxyParams}
\begin{DoxyReturn}{Returns}
If the message buffer is created successfully then a handle to the created message buffer is returned. If either puc\+Message\+Buffer\+Storage\+Area or px\+Staticmessage\+Buffer are N\+U\+LL then N\+U\+LL is returned.
\end{DoxyReturn}
Example use\+: 
\begin{DoxyPre}\end{DoxyPre}



\begin{DoxyPre}  // Used to dimension the array used to hold the messages.  The available space
  // will actually be one less than this, so 999.
#define STORAGE\_SIZE\_BYTES 1000\end{DoxyPre}



\begin{DoxyPre}  // Defines the memory that will actually hold the messages within the message
  // buffer.
  static uint8\_t ucStorageBuffer[ STORAGE\_SIZE\_BYTES ];\end{DoxyPre}



\begin{DoxyPre}  // The variable used to hold the message buffer structure.
  StaticMessageBuffer\_t xMessageBufferStruct;\end{DoxyPre}



\begin{DoxyPre}  void MyFunction( void )
  \{
  MessageBufferHandle\_t xMessageBuffer;\end{DoxyPre}



\begin{DoxyPre}   xMessageBuffer = xMessageBufferCreateStatic( sizeof( ucBufferStorage ),
                                                ucBufferStorage,
                                                \&xMessageBufferStruct );\end{DoxyPre}



\begin{DoxyPre}   // As neither the pucMessageBufferStorageArea or pxStaticMessageBuffer
   // parameters were NULL, xMessageBuffer will not be NULL, and can be used to
   // reference the created message buffer in other message buffer API calls.\end{DoxyPre}



\begin{DoxyPre}   // Other code that uses the message buffer can go here.
  \}\end{DoxyPre}



\begin{DoxyPre}  \end{DoxyPre}
 